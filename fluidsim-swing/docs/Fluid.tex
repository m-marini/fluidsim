\documentclass[a4paper,11pt]{article}
%\usepackage[T1]{fontenc}
\usepackage[utf8]{inputenc}
\usepackage{lmodern}
\usepackage[italian]{babel}
\usepackage{graphicx}
\newtheorem{theorem}{Teorema}

\title{Fluidi}
\author{Marco Marini}

\begin{document}

\maketitle
\tableofcontents

\begin{abstract}
\end{abstract}

\part{Fluidi}

\section{Approssimazione del gradiente in $ \Re^2 $}
%==========================================================


Sia $ F(\mathbf{r}) $ una funzione scalare nello spazio.

Il gradiente di $ F $ è una funzione vettoriale definita come 
\[
  \nabla F(\mathbf{r}) =
    	\frac{\partial F(\mathbf{r})}{\partial x} \mathbf{e}_x +
    	\frac{\partial F(\mathbf{r})}{\partial y} \mathbf{e}_y   	
\]

Supponiamo che il continuo sia rappresentato da un reticolo quadrato semplice dove 
\[ 
\begin{array}{c}
  \mathbf{r}(i,j) = i \Delta x \mathbf{e}_x + j \Delta y \mathbf{e}_y
  \\
    F(\mathbf{r}) = F(i,j) = F(\mathbf{r}(i,j))
 \end{array}
\]
allora sarà
\begin{equation}
  \nabla F(i,j) =
    \frac{F(i+1,j)-F(i-1,j)}{2 \Delta x} \mathbf{e}_x +
    \frac{F(i,j+1)-F(i,j-1)}{2 \Delta y} \mathbf{e}_y
\end{equation}


\section{I Approssimazione della divergenza in $ \Re^2 $}
%==========================================================

Sia $ \mathbf{F}(\mathbf{r}) $ una funzione vettoriale nello spazio.

La divergenza di $ \mathbf{F} $ è una funzione scalare definita come 
\[
  \nabla \cdot \mathbf{F}(\mathbf{r}) =
	\frac{\partial F_x(\mathbf{r})}{\partial x} +
	\frac{\partial F_y(\mathbf{r})}{\partial y}
\]

Supponiamo che il continuo sia rappresentato da un reticolo quadrato semplice dove 
\[
\begin{array}{c}
  \mathbf{r}(i,j) = i \Delta x \mathbf{e}_x + j \Delta y \mathbf{e}_y
  \\
  \mathbf{F}(\mathbf{r}) = \mathbf{F}(i,j) = \mathbf{F}(\mathbf{r}(i,j))
\end{array} 
\]
allora sarà
\begin{equation}
\label{eq:aprdiv}
  \nabla \cdot \mathbf{F}(i,j) =
    \frac{F_x(i+1,j)-F_x(i-1,j)}{2 \Delta x} +
    \frac{F_y(i,j+1)-F_y(i,j-1)}{2 \Delta y}
\end{equation}


\section{II Approssimazione della divergenza in $ \Re^2 $}
%==========================================================

Sia $ \mathbf{F}(\mathbf{r}) $ una funzione vettoriale nello spazio.

La divergenza di $ \mathbf{F} $ è una funzione scalare definita come 
\[
  \nabla \cdot \mathbf{F}(\mathbf{r})
   = \lim_{V\rightarrow p} \int \int_{S(V)} \frac{\mathbf{F} \cdot \mathbf{n}}{|V|} dS
   = \lim_{V\rightarrow p} \int \int_{S(V)}
    \frac{F_x n_x + F_y n_y}{|V|} dS
\]

Supponiamo che il continuo sia rappresentato da un reticolo dove 
le celle adiacenti siano date da $ n $ raggi vettori $ \mathbf{s}_i $
la cui superficie di contatto sia $ S_i $.
Se volume di ogni cella è $ V $ la divergenza allora sarà
\begin{equation}
  \label{eq:iiaprdiv}
  \nabla \cdot \mathbf{F}(\mathbf{r})
   = \sum_i \frac{\mathbf{F}(\mathbf{r}+\mathbf{s}_i) \cdot \mathbf{n}_i}{V} S_i
   = \sum_i \frac{F_x(\mathbf{r}+\mathbf{s}_i) n_{xi}+F_y(\mathbf{r}+\mathbf{s}_i) n_{yi}}{V} S_i
\end{equation}

Nel caso del reticolo quadrato semplice abbiamo che
\[
\begin{array}{lll}
  \mathbf{s}_1=(\Delta x,0) = \mathbf{s}_x & \mathbf{n}_1=(1,0) & S_1 = 2 \Delta y
  \\
  \mathbf{s}_2=(0, \Delta y) = \mathbf{s}_y & \mathbf{n}_2=(0,1) & S_2 = 2 \Delta x
  \\ 
  \mathbf{s}_3=-\mathbf{s}_x & \mathbf{n}_3=-\mathbf{n}_1 & S_3 = S_1
  \\
  \mathbf{s}_4=-\mathbf{s}_y & \mathbf{n}_4=\mathbf{n}_2 & S_4 = S_2
  \\
  V = 4 \Delta x \Delta y
\end{array}
\]
\begin{equation}
  \nabla \cdot \mathbf{F}(\mathbf{r}) =
    \frac{F_x(\mathbf{r} + \mathbf{s}_x) - F_x(\mathbf{r} - \mathbf{s}_x)}{2 \Delta x}+
    \frac{F_y(\mathbf{r} + \mathbf{s}_y) - F_y(\mathbf{r} - \mathbf{s}_y)}{2 \Delta y}
\end{equation}
che corrisponde con la (\ref{eq:aprdiv})



\section{Approssimazione della divergenza tensoriale in $ \Re^2 $}
%==========================================================

Sia $ \mathbf{T}(\mathbf{r}) $ una funzione tensoriale dello spazio.
\[
	\mathbf{T}(\mathbf{r}) = 
	\left|
	\begin{array}{ll}
	T_{xx}(\mathbf{r}), & T_{xy}(\mathbf{r})
	\\
	T_{yx}(\mathbf{r}), & T_{yy}(\mathbf{r})
	\end{array}
	\right|
\]

La divergenza di $ \mathbf{T} $ è una funzione vettoriale definita come 
\[
  \nabla \cdot \mathbf{T}(\mathbf{r})
    = \frac{ \partial T_{ij}(\mathbf{r})}{\partial x_i} \mathbf{e}_j
  	= \left( \frac{ \partial T_{xx}(\mathbf{r})}{\partial x} 
  	+ \frac{ \partial T_{yx}(\mathbf{r})}{\partial y} 
  	\right) \mathbf{e}_x
  	+ \left( \frac{ \partial T_{xy}(\mathbf{r})}{\partial x} 
  	+ \frac{ \partial T_{yy}(\mathbf{r})}{\partial y} 
  	\right) \mathbf{e}_y
\]

Nel caso del reticolo quadrato semplice si ha
\[
\begin{array}{r}
  \nabla \cdot \mathbf{T}(\mathbf{r})
    = 
    	\left[
    \frac{T_{xx}(\mathbf{r}+\mathbf{s}_x)-T_{xx}((\mathbf{r}-\mathbf{s}_x)}{2 \Delta x}
+    \frac{T_{yx}(\mathbf{r}+\mathbf{s}_y)-T_{yx}((\mathbf{r}-\mathbf{s}_y)}{2 \Delta y}
    \right] \mathbf{e}_x
    \\
    +
    	\left[
    \frac{T_{xy}(\mathbf{r}+\mathbf{s}_x)-T_{xy}((\mathbf{r}-\mathbf{s}_x)}{2 \Delta x}
+    \frac{T_{yy}(\mathbf{r}+\mathbf{s}_y)-T_{yy}((\mathbf{r}-\mathbf{s}_y)}{2 \Delta y}
    \right] \mathbf{e}_y
\end{array}
\]


\section{Approssimazione delle equazioni di Eulero}
%==========================================================

Definamo in ogni punto dello spazio due quantità.
\begin{itemize}
  \item
   $ \rho $ la densità di fluido in un punto
   \item
   $ \mathbf{Q} $ la quantità di moto del fluido in un punto
\end{itemize}

Partiamo dall'equazione di conservazione della massa
\[
  \frac{\partial \rho}{\partial t} = - \nabla \cdot (\rho \mathbf{u}) =
  - \nabla \cdot \mathbf{Q}
\]

Per la (\ref{eq:iiaprdiv}) abbiamo
\[
\begin{array}{r}
  \nabla \cdot \mathbf{Q} =
    \sum_i \frac{\mathbf{Q} (\mathbf{r}+\mathbf{s}_i) \cdot \mathbf{n}_i}{V} S_i
  \\
  \frac{\partial \rho}{\partial t} = - \sum_i \frac{\mathbf{Q} (\mathbf{r}+\mathbf{s}_i) \cdot \mathbf{n}_i}{V} S_i
\end{array}
\]
da cui
\[
  \Delta \rho = \frac{\partial \rho}{\partial t} \Delta t = 
  - \Delta t \sum_i \frac{\mathbf{Q} (\mathbf{r}+\mathbf{s}_i) \cdot \mathbf{n}_i}{V} S_i
\]

Per il reticolo quadrato abbiamo
\begin{equation}
	\label{eq:deltarho}
  \Delta \rho
  = - \Delta t 
  \left(
   \frac{Q_x(\mathbf{r}+\mathbf{s}_x) - Q_x(\mathbf{r}-\mathbf{s}_x)}{2 \Delta x} +
   \frac{Q_y(\mathbf{r}+\mathbf{s}_y) - Q_y(\mathbf{r}-\mathbf{s}_y)}{2 \Delta y}
   \right)
\end{equation}

Prendiamo ora l'equazione di conservazione della quantità di moto
\[
  \frac{\partial \rho \mathbf{u}}{\partial t} = \nabla \cdot \left( \rho \mathbf{u} \otimes \mathbf{u} \right) - \nabla K \rho + \rho \mathbf{g} 
\]
dove
\[
  \mathbf{u} \otimes \mathbf{u} = \left|
  \begin{array}{rrr}
    u_x u_x, & u_x u_y
    \\
    u_y u_x, & u_y u_y
  \end{array}
  \right|
\]
\[
  K = \frac{1000 R T}{P_m}
\]

$ \mathbf{g} $ è il campo gravitazionale nel punto $ \mathbf{r} $.

Calcoliamo $ \rho \mathbf{u} \otimes \mathbf{u} $
\[
	\rho \mathbf{u} \otimes \mathbf{u} = \frac{\mathbf{Q} \otimes \mathbf{Q}}{\rho}
\]
\[
  \frac{\partial \mathbf{Q}}{\partial t} =
    \nabla \cdot \left( \frac{\mathbf{Q} \otimes \mathbf{Q}}{\rho} \right) - \nabla K \rho + \rho \mathbf{g} 
\]
da cui
\[
  \Delta \mathbf{Q} = \frac{\partial \mathbf{Q}}{\partial t} \Delta t =
    \Delta t \left[ \nabla \cdot \left( \frac{\mathbf{Q} \otimes \mathbf{Q}}{\rho} \right) - \nabla K \rho + \rho \mathbf{g} \right]
\]

Espandiamo ora
\[
\begin{array}{r}
	\nabla \cdot \left( \frac{\mathbf{Q} \otimes \mathbf{Q}}{\rho} \right)
	= \left[
	\frac{\partial}{\partial x} \left( \frac{q_x q_x}{\rho} \right)
	+ \frac{\partial}{\partial y} \left( \frac{q_y q_x}{\rho} \right)
	\right] \mathbf{e}_x
	\\
	+ \left[
	\frac{\partial}{\partial x} \left( \frac{q_x q_y}{\rho} \right)
	+ \frac{\partial}{\partial y} \left( \frac{q_y q_y}{\rho} \right)
	\right] \mathbf{e}_y
\end{array}
\]

nel reticolo quadrato diventa
\[
\begin{array}{r}
	\nabla \cdot \left( \frac{\mathbf{Q} \otimes \mathbf{Q}}{\rho} \right)
	= \left[
	\frac{(q_x(\mathbf{r}+\mathbf{s}_x))^2}{2 \rho(\mathbf{r}+\mathbf{s}_x) \Delta x}
	-\frac{(q_x(\mathbf{r}-\mathbf{s}_x))^2}{2 \rho(\mathbf{r}-\mathbf{s}_x) \Delta x}
	+ \frac{q_y(\mathbf{r}+\mathbf{s}_y) q_x(\mathbf{r}+\mathbf{s}_y)}{2 \rho(\mathbf{r}+\mathbf{s}_y) \Delta y}
	- \frac{q_y(\mathbf{r}-\mathbf{s}_y) q_x(\mathbf{r}-\mathbf{s}_y)}{2 \rho(\mathbf{r}-\mathbf{s}_y) \Delta y}
	\right] \mathbf{e}_x
	\\
	+ \left[
	\frac{q_x(\mathbf{r}+\mathbf{s}_x) q_y(\mathbf{r}+\mathbf{s}_x)}{2 \rho(\mathbf{r}+\mathbf{s}_x) \Delta x}
	-\frac{q_x(\mathbf{r}-\mathbf{s}_x) q_y(\mathbf{r}-\mathbf{s}_x)}{2 \rho(\mathbf{r}-\mathbf{s}_x) \Delta x}
	+ \frac{(q_y(\mathbf{r}+\mathbf{s}_y))^2}{2 \rho(\mathbf{r}+\mathbf{s}_y) \Delta y}
	- \frac{(q_y(\mathbf{r}-\mathbf{s}_y))^2}{2 \rho(\mathbf{r}-\mathbf{s}_y) \Delta y}
	\right] \mathbf{e}_y
\end{array}
\]

Calcoliamo ora la componente $ \nabla K \rho $ nel caso del reticolo quadrato
\[
	\nabla K \rho
	= K \left[
	\frac{\rho (\mathbf{r}+\mathbf{s}_x) - \rho (\mathbf{r}-\mathbf{s}_x)}{2 \Delta x} \mathbf{e}_x
	+ \frac{\rho (\mathbf{r}+\mathbf{s}_y) - \rho (\mathbf{r}-\mathbf{s}_y)}{2 \Delta y} \mathbf{e}_y
	\right]
\]


\end{document}
