\documentclass[a4paper,11pt]{article}
\usepackage[T1]{fontenc}
\usepackage[utf8]{inputenc}
\usepackage{lmodern}
%\usepackage[italian]{babel}
\usepackage{graphicx}
\newtheorem{theorem}{Teorema}

\title{Fluidi}
\author{Marco Marini}

\begin{document}

\maketitle
\tableofcontents

\begin{abstract}
\end{abstract}

\part{Fluidi}

\section{Equazioni di Eulero dei fluidi non viscosi}

La simulazione avviene in una superficie bidimensionale (sezione).

Prendiamo una cella elementare di forma qualsiasi.
Nel caso esemplificativo un quadrato, ma nella simulazione sarà una cella esagonale.

\subsection{Equazione di continuità o bilancio di massa} 

La prima equazione è basata sul principio di continuo e di conservazione della materia

\emph{
La variazione di massa in una cella deve essere uguale alla somma dei flussi entranti nel blocco.
}
 
Iniziamo con definire la massa della cella 
\begin{equation}
    m_i = \rho_i v
\end{equation}

dove $ \rho_i $ è la densità del fluido nella cella $ C_i $ e $ v $ il relativo volume, si suppone che le celle abbiano tutte lo stesso volume. 

Adiacente alla cella $ C_i $ troviamo la cella $ C_j $, la superficie di contatto $ S_{ij} $ e il versore $ \vec{N_{ij}} $ normale alla superficie.

Definiamo $ \vec{V_i} $ la velocità di flusso passante per la cella.
  
Nell'intervallo di tempo infinitesimale $ dt $ la variazione di massa che entra verso la cella $ C_j $ è dato da  
\begin{equation}
	d m_{ij} = -( \vec{V_j} -  \vec{V_i} )\; \vec{N_{ij}} \; S_{ij} \; \rho_i \; dt
	=-( \vec{V_j} - \vec{V_i}) \; \vec{N_{ij}} \; S_{ij} \; \frac{m_i}{v} \; dt
\end{equation}

Simmetricamente
\begin{equation}
	d m_{ji} = -( \vec{V_i} - \vec{V_j}) \; \vec{N_{ji}} \; S_{ji} \; \frac{m_j}{v} \; dt
\end{equation}
essendo $ \vec{N{ji}} = -\vec{N_{ij}} $ e $ S_{ji} = S_{ij} $ 
\begin{equation}
	d m_{ji} = -( \vec{V_j} - \vec{V_i}) \; \vec{N_{ij}} \; S_{ij} \; \frac{m_j}{v} \; dt
\end{equation}

Definiamo
\begin{equation}
  K_{ij} = K_{ji} = -\frac{( \vec{V_j} - \vec{V_i}) \; \vec{N_{ij}} \; S_{ij}}{v}
\end{equation}

Si possono esprime le variazioni di massa come
\begin{eqnarray}
  d m_{ij} =   K_{ij} m_i dt
\\
  d m_{ji} =   K_{ij} m_j dt
\end{eqnarray}

In generale nella cella $ i $ avremo che
\begin{equation}
     dm_i = -\frac{m_i}{v} \sum_j \left( \vec{V_j} - \vec{V_i}\right) \; \vec{N_{ij}} \; S_{ij} \; dt = m_i \sum_j K_{ij} dt
\end{equation}


\subsection{Equazione di bilancio della quantità di moto}

La seconda equazione è basata sul principio conservazione della quantità di moto

\emph{
La variazione di quantità di moto in una cella è uguale alla somma delle quantità
di moto dei flussi entranti nel blocco più le forze agenti sul volume e sulla superficie.
}
 
Iniziamo con definire
\begin{equation}
       \vec{Q_i} = m_i \vec{V_i}
\end{equation}
la quantità di moto del fluido nella cella $ i $.

Calcoliamo la portata della quantità di moto entrante nella cella dalla cella adiacente $ j $.

Nell'intervallo di tempo infinitesimale $ dt $ la variazione di quantità di moto che entra verso la cella $ j $ è dato da  
\begin{eqnarray}
  \vec{dQ_{ij}} = dm_{ij} \vec{V_i} \; dt
  \\
  \vec{dQ_{ij}} = m_i \vec{V_i} K_{ij} \; dt^2  
\end{eqnarray}

Simmetricamente
\begin{eqnarray}
  \vec{dQ_{ji}} = m_j \vec{V_j} K_{ij} \; dt^2  
\end{eqnarray}

Se ipotiziamo un fluido non viscoso le forze applicate alla superficie $ S_{ij} $ sono perpendicolari alla superficie stessa
e dovute solo alla differenza di pressione $ P_j - P_i $ tra le due celle. L'impulso applicato alla superficie allora sarà
\begin{eqnarray}
  \vec{I_{ij}} = (P_{j} - P_{i} ) S_{ij} \vec{N_{ij}} dt 
  \\
  \vec{I_{ji}} = (P_{i} - P_{j} ) S_{ij} \vec{N_{ji}} dt = \vec{I_{ij}}
\end{eqnarray}


Mentre l'impulso causato dal campo gravitazionale $ \vec{g_i} $ sarà
\begin{equation}
  \vec{I_i} = \vec{g_i} m_i dt
\end{equation}

Applicando il principio di conservazione della quantità di moto abbiamo
\begin{equation}
   d \vec{Q_i} = \vec{g_i} m_i dt + \sum_j (P_{j} - P_{i} ) S_{ij} \vec{N_{ij}} dt
   -  \frac{m_i}{v} \vec{V_i} \sum_j \left( \vec{V_j} - \vec{V_i}\right) \; \vec{N_{ij}} \;  \; S_{ij} \; dt^2
\end{equation}

La pressione nella cella $ i $ è data da
\begin{equation}
  P_i = m_i C
\end{equation}
dove $ C = \frac{R T}{P_m v} $ con $ P_m $ peso molecolare, $ R $ costante di Avogadro e $ T $ temperatura assoluta del fluido
\end{document}
